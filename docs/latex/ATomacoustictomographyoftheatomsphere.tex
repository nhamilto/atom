%% Generated by Sphinx.
\def\sphinxdocclass{report}
\documentclass[letterpaper,10pt,english]{sphinxmanual}
\ifdefined\pdfpxdimen
   \let\sphinxpxdimen\pdfpxdimen\else\newdimen\sphinxpxdimen
\fi \sphinxpxdimen=.75bp\relax

\PassOptionsToPackage{warn}{textcomp}
\usepackage[utf8]{inputenc}
\ifdefined\DeclareUnicodeCharacter
 \ifdefined\DeclareUnicodeCharacterAsOptional
  \DeclareUnicodeCharacter{"00A0}{\nobreakspace}
  \DeclareUnicodeCharacter{"2500}{\sphinxunichar{2500}}
  \DeclareUnicodeCharacter{"2502}{\sphinxunichar{2502}}
  \DeclareUnicodeCharacter{"2514}{\sphinxunichar{2514}}
  \DeclareUnicodeCharacter{"251C}{\sphinxunichar{251C}}
  \DeclareUnicodeCharacter{"2572}{\textbackslash}
 \else
  \DeclareUnicodeCharacter{00A0}{\nobreakspace}
  \DeclareUnicodeCharacter{2500}{\sphinxunichar{2500}}
  \DeclareUnicodeCharacter{2502}{\sphinxunichar{2502}}
  \DeclareUnicodeCharacter{2514}{\sphinxunichar{2514}}
  \DeclareUnicodeCharacter{251C}{\sphinxunichar{251C}}
  \DeclareUnicodeCharacter{2572}{\textbackslash}
 \fi
\fi
\usepackage{cmap}
\usepackage[T1]{fontenc}
\usepackage{amsmath,amssymb,amstext}
\usepackage{babel}
\usepackage{times}
\usepackage[Bjarne]{fncychap}
\usepackage{sphinx}

\usepackage{geometry}

% Include hyperref last.
\usepackage{hyperref}
% Fix anchor placement for figures with captions.
\usepackage{hypcap}% it must be loaded after hyperref.
% Set up styles of URL: it should be placed after hyperref.
\urlstyle{same}
\addto\captionsenglish{\renewcommand{\contentsname}{Contents:}}

\addto\captionsenglish{\renewcommand{\figurename}{Fig.}}
\addto\captionsenglish{\renewcommand{\tablename}{Table}}
\addto\captionsenglish{\renewcommand{\literalblockname}{Listing}}

\addto\captionsenglish{\renewcommand{\literalblockcontinuedname}{continued from previous page}}
\addto\captionsenglish{\renewcommand{\literalblockcontinuesname}{continues on next page}}

\addto\extrasenglish{\def\pageautorefname{page}}

\setcounter{tocdepth}{1}



\title{ATom: acoustic tomography of the atomsphere Documentation}
\date{Sep 07, 2018}
\release{1.0}
\author{Nicholas Hamilton, James Hansen}
\newcommand{\sphinxlogo}{\vbox{}}
\renewcommand{\releasename}{Release}
\makeindex

\begin{document}

\maketitle
\sphinxtableofcontents
\phantomsection\label{\detokenize{index::doc}}



\chapter{ATom: acoustic tomography of the atomsphere}
\label{\detokenize{README:atom-acoustic-tomography-of-the-atomsphere}}\label{\detokenize{README::doc}}
ATom is a python package designed for the analysis of acoustic tomography data, including processing raw data from acoustic sources and receivers.
\begin{description}
\item[{Author: \sphinxhref{mailto:nicholas.hamilton@nrel.gov}{Nicholas Hamilton}}] \leavevmode
\sphinxhref{mailto:jj.hansen@asu.edu}{James Hansen}

\end{description}


\section{Acknowledgments}
\label{\detokenize{README:acknowledgments}}
Thanks to anyone who would like to contribute! Please see below.


\section{License}
\label{\detokenize{README:license}}
Please read LICENSE for details on the license for use and sharing.


\chapter{Code Reference}
\label{\detokenize{code:code-reference}}\label{\detokenize{code::doc}}

\section{\sphinxstyleliteralintitle{\sphinxupquote{atom\_functions}} module}
\label{\detokenize{code:atom-functions-module}}
This module contains the fundamental object classes and functions of the package.

\phantomsection\label{\detokenize{code:module-ATom.atom_functions}}\index{ATom.atom\_functions (module)}\index{butter\_bandpass() (in module ATom.atom\_functions)}

\begin{fulllineitems}
\phantomsection\label{\detokenize{code:ATom.atom_functions.butter_bandpass}}\pysiglinewithargsret{\sphinxcode{\sphinxupquote{ATom.atom\_functions.}}\sphinxbfcode{\sphinxupquote{butter\_bandpass}}}{\emph{lowcut}, \emph{highcut}, \emph{fs}, \emph{order=5}}{}
create band-pass frequency filter to isolate chirp frequency in mic singals
\begin{description}
\item[{Parameters:}] \leavevmode\begin{description}
\item[{lowcut: float}] \leavevmode
lower limit of frequency band

\item[{highcut: float}] \leavevmode
upper limit of frequency band

\item[{fs: float}] \leavevmode
sampling frequency of data

\item[{order: int}] \leavevmode
filter order, default = 5

\end{description}

\item[{Returns:}] \leavevmode\begin{description}
\item[{b, a: ndarray, ndarray}] \leavevmode
Numerator (b) and denominator (a) polynomials of the IIR filter. Only returned if output=’ba’.

\end{description}

\end{description}

\end{fulllineitems}

\index{butter\_bandpass\_filter() (in module ATom.atom\_functions)}

\begin{fulllineitems}
\phantomsection\label{\detokenize{code:ATom.atom_functions.butter_bandpass_filter}}\pysiglinewithargsret{\sphinxcode{\sphinxupquote{ATom.atom\_functions.}}\sphinxbfcode{\sphinxupquote{butter\_bandpass\_filter}}}{\emph{data}, \emph{lowcut}, \emph{highcut}, \emph{fs}, \emph{order=5}}{}
implement fileter on input data
\begin{description}
\item[{Parameters:}] \leavevmode\begin{description}
\item[{data: np.arary}] \leavevmode
acoustic signal to be filtered (microphone data)

\item[{lowcut: float}] \leavevmode
lower limit of frequency band, passed to \sphinxtitleref{butter\_bandpass}

\item[{highcut: float}] \leavevmode
upper limit of frequency band, passed to \sphinxtitleref{butter\_bandpass}

\item[{fs: float}] \leavevmode
sampling frequency of data, passed to \sphinxtitleref{butter\_bandpass}

\item[{order: int}] \leavevmode
filter order, default = 5, passed to \sphinxtitleref{butter\_bandpass}

\end{description}

\item[{Returns:}] \leavevmode\begin{description}
\item[{y: np.array}] \leavevmode
frequency-filtered data

\end{description}

\end{description}

\end{fulllineitems}

\index{covariance() (in module ATom.atom\_functions)}

\begin{fulllineitems}
\phantomsection\label{\detokenize{code:ATom.atom_functions.covariance}}\pysiglinewithargsret{\sphinxcode{\sphinxupquote{ATom.atom\_functions.}}\sphinxbfcode{\sphinxupquote{covariance}}}{\emph{micdat}, \emph{speakerdat}}{}
Lag-N cross correlation between two signals.
Only the correlation between a speaker chirp and its respective
signal in each microphone record sample is required.
\begin{description}
\item[{Parameters:}] \leavevmode\begin{description}
\item[{micdat: pd.DataFrame}] \leavevmode
extracted microphone data containing received acoustic signals

\item[{speakerdat: pd.DataFrame}] \leavevmode
extracted speaeker acoustic signals

\end{description}

\item[{Returns}] \leavevmode\begin{description}
\item[{covar: np.array}] \leavevmode
time-lag correlation between micdat and speakerdat

\end{description}

\end{description}

\end{fulllineitems}

\index{covariance\_depricated() (in module ATom.atom\_functions)}

\begin{fulllineitems}
\phantomsection\label{\detokenize{code:ATom.atom_functions.covariance_depricated}}\pysiglinewithargsret{\sphinxcode{\sphinxupquote{ATom.atom\_functions.}}\sphinxbfcode{\sphinxupquote{covariance\_depricated}}}{\emph{micdat}, \emph{speakerdat}}{}
Lag-N cross correlation.
Parameters
\begin{quote}
\begin{description}
\item[{micdat: pd.DataFrame}] \leavevmode
extracted microphone data containing received acoustic signals

\item[{speakerdat: pd.DataFrame}] \leavevmode
extracted speaeker acoustic signals

\end{description}
\end{quote}
\begin{description}
\item[{Returns}] \leavevmode
covar: float

\end{description}

\end{fulllineitems}

\index{crosscorr\_depricated() (in module ATom.atom\_functions)}

\begin{fulllineitems}
\phantomsection\label{\detokenize{code:ATom.atom_functions.crosscorr_depricated}}\pysiglinewithargsret{\sphinxcode{\sphinxupquote{ATom.atom\_functions.}}\sphinxbfcode{\sphinxupquote{crosscorr\_depricated}}}{\emph{datax}, \emph{datay}, \emph{lag=0}}{}
Lag-N cross correlation.
\begin{description}
\item[{Parameters}] \leavevmode\begin{description}
\item[{lag: int}] \leavevmode
default 0

\item[{datax, datay: pandas.Series}] \leavevmode
objects of equal length

\end{description}

\item[{Returns}] \leavevmode
covar: float

\end{description}

\end{fulllineitems}

\index{dataset (class in ATom.atom\_functions)}

\begin{fulllineitems}
\phantomsection\label{\detokenize{code:ATom.atom_functions.dataset}}\pysiglinewithargsret{\sphinxbfcode{\sphinxupquote{class }}\sphinxcode{\sphinxupquote{ATom.atom\_functions.}}\sphinxbfcode{\sphinxupquote{dataset}}}{\emph{datapath}}{}
dataset is the object class for raw data. It should contain a directory in where
raw data are to be found, data I/O routines, experiement constants, array
calibration info, etc.
\index{extract\_travel\_times() (ATom.atom\_functions.dataset method)}

\begin{fulllineitems}
\phantomsection\label{\detokenize{code:ATom.atom_functions.dataset.extract_travel_times}}\pysiglinewithargsret{\sphinxbfcode{\sphinxupquote{extract\_travel\_times}}}{\emph{upsamplefactor=10}, \emph{searchLag=None}, \emph{filterflag=True}, \emph{verbose=False}}{}
Main processing step of raw data.

Acoustic chirps are identified in speaker and microphone signals.
Travel time from each speaker to each mic are calculated.
\begin{description}
\item[{Parameters:}] \leavevmode\begin{description}
\item[{upsamplefactor: int}] \leavevmode
degree to which acoustic signals are upsampled. This is needed to
increase precision of travel time estimate

\item[{searchLag: int}] \leavevmode
acoustic signal window width. If none is provided, a default window
width is assigned of \sphinxtitleref{searchLag = 3 * self.meta.chirp\_record\_length * upsamplefactor}

\item[{filterflag: bool}] \leavevmode
implement frequency filter to microphone signals to remove spurious
spectral contributions. Band-pass filter with acoustic chirp bandwidth
around the central frequency of the acoustic chip, with the bandwidth

\item[{verbose: bool}] \leavevmode
determine output text. used to debug.

\end{description}

\item[{Returns:}] \leavevmode\begin{description}
\item[{ATom\_signals: np.ndarray {[}nspeakers, nmics, searchLag, nrecords{]}}] \leavevmode
acoustic chirps received by the microphones

\item[{travel\_times: np.ndarray {[}nspeakers, nmics, nrecords{]}}] \leavevmode
travel times (ms) of chirps between each speaker and mic for each record

\item[{travel\_inds: np.ndarray {[}nspeakers, nmics, nrecords{]}}] \leavevmode
travel times (samples) of chirps between each speaker and mic for each record

\end{description}

\end{description}

\end{fulllineitems}

\index{get\_calibration\_info() (ATom.atom\_functions.dataset method)}

\begin{fulllineitems}
\phantomsection\label{\detokenize{code:ATom.atom_functions.dataset.get_calibration_info}}\pysiglinewithargsret{\sphinxbfcode{\sphinxupquote{get\_calibration\_info}}}{\emph{caldatapath}}{}
get locations of speakers and mics, signal latency between particular devices,
and sound propagation delays from speakers as a function of azimuth
\begin{description}
\item[{Parameters:}] \leavevmode\begin{description}
\item[{caldatapath: str}] \leavevmode\begin{description}
\item[{path to directory containing raw data}] \leavevmode\begin{itemize}
\item {} 
‘average\_latency\_yymmdd.csv’

\item {} 
‘mic\_locations\_yymmdd.csv’

\item {} 
‘speaker\_locations\_yymmdd.csv’

\end{itemize}

\item[{and/or containing processed data to import}] \leavevmode\begin{itemize}
\item {} 
‘offsets.py’

\end{itemize}

\end{description}

\end{description}

\end{description}

\end{fulllineitems}

\index{get\_constants() (ATom.atom\_functions.dataset method)}

\begin{fulllineitems}
\phantomsection\label{\detokenize{code:ATom.atom_functions.dataset.get_constants}}\pysiglinewithargsret{\sphinxbfcode{\sphinxupquote{get\_constants}}}{\emph{constantspath}}{}
get values of constants used in experiment

Parameters:
\begin{quote}
\begin{description}
\item[{constantspath: str}] \leavevmode
path to directory containing ‘constants.py’

\end{description}
\end{quote}

\end{fulllineitems}

\index{get\_meta() (ATom.atom\_functions.dataset method)}

\begin{fulllineitems}
\phantomsection\label{\detokenize{code:ATom.atom_functions.dataset.get_meta}}\pysiglinewithargsret{\sphinxbfcode{\sphinxupquote{get\_meta}}}{\emph{constantspath}}{}
get meta data for experiment

Parameters:
\begin{quote}

constantspath: directory path to file containing meta data
\end{quote}

\end{fulllineitems}

\index{load\_aux() (ATom.atom\_functions.dataset method)}

\begin{fulllineitems}
\phantomsection\label{\detokenize{code:ATom.atom_functions.dataset.load_aux}}\pysiglinewithargsret{\sphinxbfcode{\sphinxupquote{load\_aux}}}{\emph{sampletime}}{}
load data file into dataset object
\begin{description}
\item[{Parameters:}] \leavevmode\begin{description}
\item[{aux\_dataapath: str}] \leavevmode
path to directory containing aux data

\end{description}

\end{description}

\end{fulllineitems}

\index{load\_data\_sample() (ATom.atom\_functions.dataset method)}

\begin{fulllineitems}
\phantomsection\label{\detokenize{code:ATom.atom_functions.dataset.load_data_sample}}\pysiglinewithargsret{\sphinxbfcode{\sphinxupquote{load\_data\_sample}}}{\emph{fileID}}{}
load data file into dataset object
\begin{description}
\item[{Parameters:}] \leavevmode\begin{description}
\item[{fileID: int}] \leavevmode
integer value of main and aux data in respective lists

\end{description}

\end{description}

\end{fulllineitems}

\index{load\_main() (ATom.atom\_functions.dataset method)}

\begin{fulllineitems}
\phantomsection\label{\detokenize{code:ATom.atom_functions.dataset.load_main}}\pysiglinewithargsret{\sphinxbfcode{\sphinxupquote{load\_main}}}{\emph{sampletime}}{}
load data file into dataset object
\begin{description}
\item[{Parameters:}] \leavevmode\begin{description}
\item[{maindatapath: str}] \leavevmode
path to directory containing main data

\end{description}

\end{description}

\end{fulllineitems}

\index{signal\_ETAs() (ATom.atom\_functions.dataset method)}

\begin{fulllineitems}
\phantomsection\label{\detokenize{code:ATom.atom_functions.dataset.signal_ETAs}}\pysiglinewithargsret{\sphinxbfcode{\sphinxupquote{signal\_ETAs}}}{}{}
imports the current speaker (i) and microphone (j) as well as the temperature
array for the current signal period

\end{fulllineitems}

\index{time\_info() (ATom.atom\_functions.dataset method)}

\begin{fulllineitems}
\phantomsection\label{\detokenize{code:ATom.atom_functions.dataset.time_info}}\pysiglinewithargsret{\sphinxbfcode{\sphinxupquote{time\_info}}}{}{}
print time resolution of main and aux data

\end{fulllineitems}


\end{fulllineitems}

\index{freq\_filter() (in module ATom.atom\_functions)}

\begin{fulllineitems}
\phantomsection\label{\detokenize{code:ATom.atom_functions.freq_filter}}\pysiglinewithargsret{\sphinxcode{\sphinxupquote{ATom.atom\_functions.}}\sphinxbfcode{\sphinxupquote{freq\_filter}}}{\emph{datasample}, \emph{filter\_freq\_inds}}{}
frequency filter to isolate chirp signal in microphones
brute force method takes FFT of datasample, sets frequecies
outside specified windows to zero, implements IFFT.

Probably produces ringing in data. Should probably use proper filter.
\begin{description}
\item[{Parameters:}] \leavevmode\begin{description}
\item[{datasample: np.array}] \leavevmode
data to filter

\item[{filter\_freq\_inds: np.array}] \leavevmode
key frequencies used in filter design.

\end{description}

\end{description}

\end{fulllineitems}

\index{get\_speaker\_signal\_delay() (in module ATom.atom\_functions)}

\begin{fulllineitems}
\phantomsection\label{\detokenize{code:ATom.atom_functions.get_speaker_signal_delay}}\pysiglinewithargsret{\sphinxcode{\sphinxupquote{ATom.atom\_functions.}}\sphinxbfcode{\sphinxupquote{get\_speaker\_signal\_delay}}}{\emph{speakersamp}}{}
extract the speaker signal delays from a single record
\begin{description}
\item[{Parameters}] \leavevmode\begin{description}
\item[{speakersamp: pd.DataFrame}] \leavevmode
speaker time series data for a single record

\end{description}

\item[{Returns:}] \leavevmode\begin{description}
\item[{speaker\_signal\_delay: np.array}] \leavevmode
index corresponding to detected speaker signal delays

\end{description}

\end{description}

\end{fulllineitems}

\index{meta\_data (class in ATom.atom\_functions)}

\begin{fulllineitems}
\phantomsection\label{\detokenize{code:ATom.atom_functions.meta_data}}\pysigline{\sphinxbfcode{\sphinxupquote{class }}\sphinxcode{\sphinxupquote{ATom.atom\_functions.}}\sphinxbfcode{\sphinxupquote{meta\_data}}}
Base class for instrument or data record meta data.
Takes in a list of parameters and values.
\index{from\_data() (ATom.atom\_functions.meta\_data method)}

\begin{fulllineitems}
\phantomsection\label{\detokenize{code:ATom.atom_functions.meta_data.from_data}}\pysiglinewithargsret{\sphinxbfcode{\sphinxupquote{from\_data}}}{\emph{data}}{}
pass data from constants file,
create attribute/value pairs
use to create meta data attributes

\end{fulllineitems}

\index{from\_file() (ATom.atom\_functions.meta\_data method)}

\begin{fulllineitems}
\phantomsection\label{\detokenize{code:ATom.atom_functions.meta_data.from_file}}\pysiglinewithargsret{\sphinxbfcode{\sphinxupquote{from\_file}}}{\emph{keys}, \emph{values}}{}
instantiate meta\_data object from lists of file pairs

\end{fulllineitems}

\index{from\_lists() (ATom.atom\_functions.meta\_data method)}

\begin{fulllineitems}
\phantomsection\label{\detokenize{code:ATom.atom_functions.meta_data.from_lists}}\pysiglinewithargsret{\sphinxbfcode{\sphinxupquote{from\_lists}}}{\emph{keys}, \emph{values}}{}
instantiate meta\_data object from lists of attribute/value pairs

\end{fulllineitems}

\index{to\_list() (ATom.atom\_functions.meta\_data method)}

\begin{fulllineitems}
\phantomsection\label{\detokenize{code:ATom.atom_functions.meta_data.to_list}}\pysiglinewithargsret{\sphinxbfcode{\sphinxupquote{to\_list}}}{}{}
pretty print list of attributes

\end{fulllineitems}


\end{fulllineitems}

\index{rollchannel() (in module ATom.atom\_functions)}

\begin{fulllineitems}
\phantomsection\label{\detokenize{code:ATom.atom_functions.rollchannel}}\pysiglinewithargsret{\sphinxcode{\sphinxupquote{ATom.atom\_functions.}}\sphinxbfcode{\sphinxupquote{rollchannel}}}{\emph{data}, \emph{rollval}}{}
shifts values of data forward by rollval index
\begin{description}
\item[{Parameters:}] \leavevmode\begin{description}
\item[{data: pandas.Series}] \leavevmode
data to shift in time

\item[{rollval: int}] \leavevmode
default 0

\end{description}

\end{description}

\end{fulllineitems}

\index{signalOnMic() (in module ATom.atom\_functions)}

\begin{fulllineitems}
\phantomsection\label{\detokenize{code:ATom.atom_functions.signalOnMic}}\pysiglinewithargsret{\sphinxcode{\sphinxupquote{ATom.atom\_functions.}}\sphinxbfcode{\sphinxupquote{signalOnMic}}}{\emph{micsamp}, \emph{speakersigs}, \emph{signalETAs}, \emph{searchLag}, \emph{chirp\_record\_length}}{}
Extract chirps from the microphones. Each microphone receives
(nominally 8) chirps emitted by each speaker. Known speaker/mic locations,
along with known speaker chirp emission times, together provide
expected travel times from each speaker to each mic. Time correlation
between signals determines the precise time of chirp arrival and adds
to the expected signal travel times.
\begin{description}
\item[{Parameters:}] \leavevmode\begin{description}
\item[{micsamp: pd.DataFrame}] \leavevmode
microphone signals for a single record

\item[{speakersigs: pd.DataFrame}] \leavevmode
extracted acoustic chirps from ‘signalOnSpeaker’

\item[{searchLag: int}] \leavevmode
length of search window in samples

\item[{chirp\_record\_length: int}] \leavevmode
length of acoustic chirp in samples
base value = 116
multiplied by upsample factor

\end{description}

\item[{Returns:}] \leavevmode\begin{description}
\item[{micsigs: pd.DataFrame}] \leavevmode
extracted speaker chirps, centered in a window of length searchLag
should be nSpeaker signals for each microphone

\item[{time\_received\_record: np.array}] \leavevmode
transit time of each acoustic signal in samples

\end{description}

\end{description}

\end{fulllineitems}

\index{signalOnSpeaker() (in module ATom.atom\_functions)}

\begin{fulllineitems}
\phantomsection\label{\detokenize{code:ATom.atom_functions.signalOnSpeaker}}\pysiglinewithargsret{\sphinxcode{\sphinxupquote{ATom.atom\_functions.}}\sphinxbfcode{\sphinxupquote{signalOnSpeaker}}}{\emph{speakersamp}, \emph{searchLag}, \emph{chirp\_record\_length}, \emph{speaker\_signal\_delay}}{}
Extract chirps from the speakers. These are generated signals, and
so are clean, consistent, and spaced by known amounts. Speaker
signals are compared against microphone signals to determine the
actual trasit time of acoustic chirps across the array.
\begin{description}
\item[{Parameters:}] \leavevmode\begin{description}
\item[{speakersamp: pd.DataFrame}] \leavevmode
speaker signals for a single record

\item[{searchLag: int}] \leavevmode
length of search window in samples

\item[{chirp\_record\_length: int}] \leavevmode
length of acoustic chirp in samples
base value = 116
multiplied by upsample factor

\item[{speaker\_signal\_delay: array}] \leavevmode
each speaker chirp is delayed by a specified amount
to offset the chirps in time
base values = {[}2480, 2080, 4080,    0, 3200, 4000,  800, 2880{]}
multiplied by upsample factor

\end{description}

\item[{Returns:}] \leavevmode\begin{description}
\item[{speakersigs: pd.DataFrame}] \leavevmode
extracted speaker chirps, centered in a window of length searchLag

\end{description}

\end{description}

\end{fulllineitems}

\index{upsample() (in module ATom.atom\_functions)}

\begin{fulllineitems}
\phantomsection\label{\detokenize{code:ATom.atom_functions.upsample}}\pysiglinewithargsret{\sphinxcode{\sphinxupquote{ATom.atom\_functions.}}\sphinxbfcode{\sphinxupquote{upsample}}}{\emph{datasample}, \emph{upsamplefactor}, \emph{method='cubic'}}{}
upsample data by desired factor using specified method
\begin{description}
\item[{Parameters:}] \leavevmode\begin{description}
\item[{datasample: pd.DataFrame}] \leavevmode
microphone or speaker data to upsample

\item[{upsamplefactor: int, float}] \leavevmode
factor by which to upsample data
upsamplefactor \textgreater{} 1 ==\textgreater{} increase in time resolution
upsamplefactor \textless{} 1 ==\textgreater{} decrease in time resolution

\item[{method: str}] \leavevmode
method by which to interpolate data:
\begin{itemize}
\item {} 
pandas - built-in interp method, slow

\item {} 
linear - linear interpolation

\item {} 
cubic (default) - cubic interpolation

\end{itemize}

\end{description}

\item[{Returns:}] \leavevmode\begin{description}
\item[{newdatasample: pd.DataFrame}] \leavevmode
upsampled data

\end{description}

\end{description}

\end{fulllineitems}

\index{upsample\_data\_deptricated() (in module ATom.atom\_functions)}

\begin{fulllineitems}
\phantomsection\label{\detokenize{code:ATom.atom_functions.upsample_data_deptricated}}\pysiglinewithargsret{\sphinxcode{\sphinxupquote{ATom.atom\_functions.}}\sphinxbfcode{\sphinxupquote{upsample\_data\_deptricated}}}{\emph{self}, \emph{upsamplefactor}}{}
artificially upsample data to provide the desired resolution
new\_timedelta = oldtimedelta / upsamplefactor
upsamplefactor \textgreater{} 1 ==\textgreater{} increase in time resolution
upsamplefactor \textless{} 1 ==\textgreater{} decrease in time resolution
\begin{description}
\item[{Parameters:}] \leavevmode\begin{description}
\item[{upsamplefactor: float or int}] \leavevmode
scale by which to resample data

\end{description}

\end{description}

\end{fulllineitems}



\chapter{Indices and tables}
\label{\detokenize{index:indices-and-tables}}\begin{itemize}
\item {} 
\DUrole{xref,std,std-ref}{genindex}

\item {} 
\DUrole{xref,std,std-ref}{modindex}

\item {} 
\DUrole{xref,std,std-ref}{search}

\end{itemize}


\renewcommand{\indexname}{Python Module Index}
\begin{sphinxtheindex}
\def\bigletter#1{{\Large\sffamily#1}\nopagebreak\vspace{1mm}}
\bigletter{a}
\item {\sphinxstyleindexentry{ATom.atom\_functions}}\sphinxstyleindexpageref{code:\detokenize{module-ATom.atom_functions}}
\end{sphinxtheindex}

\renewcommand{\indexname}{Index}
\printindex
\end{document}